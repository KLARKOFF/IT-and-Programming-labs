\documentclass[a4paper, 17pt]{extarticle}
% page margin
\usepackage[margin=2cm]{geometry}
% to use russian language
\usepackage[utf8]{inputenc}
\usepackage[russian]{babel}
% to write math
\usepackage{amsmath, amsthm, amssymb}
% to use links
\usepackage[colorlinks=true, linkcolor=red]{hyperref}
% to use "" properly
\usepackage{csquotes}
% for work with pictures
\usepackage{float}
\usepackage{wrapfig}
\usepackage[thinlines]{easytable}
\usepackage{subcaption}
\usepackage{graphicx}
\graphicspath{ {./pics/} }

\begin{document}
\begin{titlepage}
  \begin{center}
    \small{ \bfseries{
    МИНИСТЕРСТВО ЦИФРОВОГО \\ РАЗВИТИЯ, СВЯЗИ И \\ МАССОВЫХ
    КОММУНИКАЦИЙ \\ РОССИЙСКОЙ ФЕДЕРАГИИ \\ Ордена Трудового Красного
    Знамени федеральное государственное бюджетное учреждение высшего
    образования \enquote{Московский технический университет связи и
    информатики}}\\
  \mdseries}

    \vspace{1.7cm}

    \normalsize{Кафедра \enquote{Информационные технологии}} \\
    \normalsize{Предмет \enquote{Математические Основы Баз Данных}}

    \vspace{0.3cm}
    \huge{Лабораторная работа №4} \\
    \large{\textbf{Обработка исключений}} \\
    \vspace{0.3cm}
    \normalsize{\textit{Вариант 22}}

    \vspace{2.7cm}

    \raggedleft 
    \normalsize{Выполнил: \\ студент гр. БПИ2402 
    \\\vspace{0.125cm} Поляков Н.А.\\}
    \centering

    \vspace{\fill}

    Москва \\ 2025

  \end{center}
\end{titlepage}

\tableofcontents
\pagebreak

\section{Цель работы}

Научиться работать с ошибками в ходе выполнения кода. Освоить для этого оператор
try-catch и все его особенности.

\section{Индивидуальное задание}

\begin{figure}[h!]
\includegraphics[width=\textwidth]{задание_1.jpg}
\end{figure}

\begin{figure}[h!]
\includegraphics[width=\textwidth]{задание_2.jpg}
\end{figure}

\begin{figure}[h!]
\includegraphics[width=\textwidth]{вариант.jpg}
\end{figure}

\clearpage

\section{Выполнение}

\subsection{Задание 1}

Класс CustomDivisionException наследует класс Exception. 
Для этого простого задания достаточно лишь написать в нём конструктор.
Если бы мы хотели более продвинутую работу с ошибками, в этом классе
можно было бы, например, хранить величину делимого.

\begin{figure}[h!]
\includegraphics[width=\textwidth]{cde_class.png}
\caption{}
\end{figure}

\clearpage

Использование нового класса тоже элементарно: участок кода, где
может произойти деление на ноль, нужно засунуть в блок try-catch и
обработать данную ошибку. Вот полная программа:

\begin{figure}[h!]
\includegraphics[width=\textwidth]{cde_class_main.png}
\caption{}
\end{figure}

\clearpage

\subsection{Задание 2}

Для начала создадим классы для кастомных ошибок:

\begin{figure}[h!]
\includegraphics[width=\textwidth]{task2_classi.png}
\caption{}
\end{figure}

Далее копируем файл в блоке try, где именно будет обрабатывать ошибки,
связанные с открытием:

\begin{figure}[h!]
\includegraphics[width=\textwidth]{task2_1.png}
\caption{}
\end{figure}

В этом же блоке создаём ещё один try, только теперь обрабатываем
ошибки, связанные с записью:

\begin{figure}[h!]
\includegraphics[width=\textwidth]{task2_2.png}
\caption{}
\end{figure}

\clearpage

\subsection{Задание 3}

Сначала создаём кастомный класс для ошибки неправильного
ввода. В нём будем хранить сообщение об ошибке и сам неправильный
ввод:

\begin{figure}[h!]
\includegraphics[width=\textwidth]{task3_1.png}
\caption{}
\end{figure}

Вот так он работает:

\begin{figure}[h!]
\includegraphics[width=\textwidth]{task3_2.png}
\caption{}
\end{figure}

\section{Вывод}
Я научился обрабатывать ошибки на языке Java, а также создавать
свои собственные классы ошибок.

\section{Github}
https://github.com/KLARKOFF/IT-and-Programming-labs


\end{document}
