\documentclass[a4paper, 17pt]{extarticle}
% page margin
\usepackage[margin=2cm]{geometry}
% to use russian language
\usepackage[utf8]{inputenc}
\usepackage[russian]{babel}
% to write math
\usepackage{amsmath, amsthm, amssymb}
% to use links
\usepackage[colorlinks=true, linkcolor=red]{hyperref}
% to use "" properly
\usepackage{csquotes}
% for work with pictures
\usepackage{float}
\usepackage{wrapfig}
\usepackage[thinlines]{easytable}
\usepackage{subcaption}
\usepackage{graphicx}
\graphicspath{ {./pics/} }

\begin{document}
\begin{titlepage}
  \begin{center}
    \small{ \bfseries{
    МИНИСТЕРСТВО ЦИФРОВОГО \\ РАЗВИТИЯ, СВЯЗИ И \\ МАССОВЫХ
    КОММУНИКАЦИЙ \\ РОССИЙСКОЙ ФЕДЕРАГИИ \\ Ордена Трудового Красного
    Знамени федеральное государственное бюджетное учреждение высшего
    образования \enquote{Московский технический университет связи и
    информатики}}\\
  \mdseries}

    \vspace{1.7cm}

    \normalsize{Кафедра \enquote{Информационные технологии}} \\
    \normalsize{Предмет \enquote{Математические Основы Баз Данных}}

    \vspace{0.3cm}
    \huge{Лабораторная работа №3} \\
    \large{\textbf{Создание хэш таблицы и работа с HashMap}} \\
    \vspace{0.3cm}
    \normalsize{\textit{Вариант 2}}

    \vspace{2.7cm}

    \raggedleft 
    \normalsize{Выполнил: \\ студент гр. БПИ2402 
    \\\vspace{0.125cm} Поляков Н.А.\\}
    \centering

    \vspace{\fill}

    Москва \\ 2025

  \end{center}
\end{titlepage}

\tableofcontents
\pagebreak

\section{Цель работы}

Освоить работу с хэш-таблицами в Java. Научиться писать тип HashTable
самостоятельно с помощью метода цепочек

\section{Индивидуальное задание}

\begin{figure}[h!]
\includegraphics[width=\textwidth]{задание1.png}
\caption{}
\end{figure}

\begin{figure}[h!]
\includegraphics[width=\textwidth]{задание2.png}
\caption{Задание 2. Работа с HashMap}
\end{figure}

\section{Выполнение}

\subsection{Задание 1}

Сначала я создал класс HashTable. В примере из листинга 3.3 использовались
типы дженерики K и V для типов ключа и значения.

\begin{figure}[h!]
\includegraphics[width=\textwidth]{hashtable1.png}
\caption{Класс HashTable}
\end{figure}

\pagebreak

Далее создадим методы hash, put, get и remove:

\begin{figure}[h!]
\includegraphics[width=\textwidth]{hash.png}
\caption{метод hash}
\end{figure}

\begin{figure}[h!]
\includegraphics[width=\textwidth]{put.png}
\caption{метод put}
\end{figure}

\begin{figure}[h!]
\includegraphics[width=\textwidth]{get.png}
\caption{метод get}
\end{figure}

\pagebreak

\begin{figure}[h!]
\centering
\includegraphics[width=.7\textwidth]{remove.png}
\caption{метод remove}
\end{figure} 
Дополнительно, по заданию, реализуем методы isEmpty, getSize:
\begin{figure}[h!]
  \includegraphics[width=\textwidth]{getSizeandempty.png}
\caption{}
\end{figure}

Для удобства вывода всех значений я написал метод show, который
показывает их:
\begin{figure}[h!]
\includegraphics[width=\textwidth]{show.png}
\caption{}
\end{figure}

\pagebreak

Теперь напишем тестовую программу, где добавим несколько значений в разные
хэш-таблицы:
\begin{figure}[h!]
\includegraphics[width=\textwidth]{main111.png}
\caption{}
\end{figure}
Здесь я сначала создаю таблицу с ключом-строкой и значением-числом, записываю
в него случайные данные. Далее удаляю и смотрю, как будут меняться значения.

После я создал таблицу isPrimeTable с ключом-числом и значением-числом. В нём
я храню все числа от 0 до 999 и храню информацию о том, являются ли они простыми.
Метод isPrime я взял из лабораторной работы 

\subsection{Задание 2}

Для выполнения второго задания я для начала создал класс Product, в котором
будет храниться вся информация о конкретном продукте. А также класс 
ProductManager, который будет хранить в себе хэш-таблицу с ключом-строкой
и значением-объект класса product. Это распространённая практика управлением
данными в Java:

\begin{figure}[h!]
\includegraphics[width=\textwidth]{productclass.png}
\caption{класс Product}
\end{figure}
\begin{figure}[h!]
\includegraphics[width=\textwidth]{productmanagerclass.png}
\caption{класс ProductManager}
\end{figure}

\pagebreak

Всё! Теперь напишем простую программу, в которой будем использовать
все методы ProductManager:

\begin{figure}[h!]
\includegraphics[width=\textwidth]{main222.png}
\caption{Пример работы с ProductManager}
\end{figure}

\pagebreak

\section{Вывод}

Я освоил работу с хэш-таблица в Java, научился создавать свои собственные
таблицы с помощью метода цепочек. А также дополнительно обучился работе с
дженериками.

\section{Github}
https://github.com/KLARKOFF/IT-and-Programming-labs


\end{document}
